\documentclass{article}
\usepackage[utf8]{inputenc}
\usepackage{amsmath}
\usepackage{geometry}
\usepackage{listings}
\usepackage{xcolor}

% Define colors for the code
\definecolor{codegreen}{rgb}{0,0.6,0}
\definecolor{codegray}{rgb}{0.5,0.5,0.5}
\definecolor{codepurple}{rgb}{0.58,0,0.82}
\definecolor{backcolour}{rgb}{0.95,0.95,0.92}

% Define the style for the code listing
\lstdefinestyle{pythonstyle}{
    backgroundcolor=\color{backcolour},
    commentstyle=\color{codegreen},
    keywordstyle=\color{codepurple},
    numberstyle=\tiny\color{codegray},
    stringstyle=\color{blue},
    basicstyle=\ttfamily\small,
    breakatwhitespace=false,
    breaklines=true,
    captionpos=b,
    keepspaces=true,
    numbers=left,
    numbersep=5pt,
    showspaces=false,
    showstringspaces=false,
    showtabs=false,
    tabsize=2
}

\geometry{a4paper, margin=1in}

\title{Exercise 3}
\author{Gormery K. Wanjiru}
\date{\today}

\begin{document}

\maketitle
\section*{Problem 1}

\textbf{a) A signal \(x_1(t) = \cos(2 \pi f_0 t)\) where \(f_0 = 125\) Hz, is sampled with spacing \(T = 0.5\) ms between the samples, and the first sample is taken at \(t = 0\). Suppose we calculate a 128-point DFT of the first 128 samples of \(x_1(t)\). For what values of \(k\), will \(X(k)\) be different from 0?}

The formula for the Discrete Fourier Transform (DFT) is given by:
\[ X(k) = \sum_{n=0}^{127} \cos(2 \pi f_0 n T) \cdot e^{-j \frac{2\pi}{128}kn} \]
Values of \(k\) for which \(X(k)\) is different from 0 are those that satisfy \(f_0 n T = \frac{k}{128}\).

Given parameters: \(f_0 = 125\) Hz, \(T = 0.5\) ms = 0.0005 s.

Substitute into the equation: \(f_0 n T = \frac{k}{128}\).
\[ 125 \times n \times 0.0005 = \frac{k}{128} \]

Solve for \(k\):
\[ n = \frac{k}{(125 \times 0.0005) \times 128} \]
\[ n = \frac{k}{0.016} \]
\[ k = 0.016n \]

Since \(k\) must be an integer, let \(n\) take integer values. Therefore,
to get this numbers this student  used the following simple python code:
\begin{lstlisting}[language=Python, style=pythonstyle, caption={print the array of k values}]
    # Given parameters
    f0 = 125  # Hz
    T = 0.0005  # s
    N = 128
    
    # Solve for k
    k_values = [n * N // (f0 * T * 128) for n in range(N)]
    
    print(k_values)
\end{lstlisting}
These values represent the indices in the DFT where the corresponding \(X(k)\) will be different from 0.\\
\(k\) can take on integer values \(0, 16, 32, \ldots, 2016\)



\textbf{b) What is the frequency resolution, i.e., the frequency distance between the calculated values in a 128-point DFT? Suppose the sequence is sampled with 2000 Hz.}

The frequency resolution in a 128-point DFT is given by:
\[ \text{Frequency Resolution} = \frac{\text{Sampling Frequency}}{\text{Number of Points in DFT}} \]
Suppose the sequence is sampled with 2000 Hz.

Calculate the frequency resolution:
\[ \text{Frequency Resolution} = \frac{2000}{128} \text{ Hz} = 15,625 \text{ Hz} \]

\textbf{c) A signal is sampled with a 2000 Hz sampling frequency. How long a section of the signal (measured in time) must be analyzed to determine the frequency content with better than 1 Hz frequency resolution?}

To determine the frequency content with better than 1 Hz resolution when sampled with a 2000 Hz frequency, the time duration \(T_{\text{analyze}}\) is given by:
\[ T_{\text{analyze}} = \frac{1}{\text{Frequency Resolution}} \]

Calculate \(T_{\text{analyze}}\):
\[ T_{\text{analyze}} = \frac{1}{15,625} \approx 0.000064 \text{ s} \]

\section*{Problem 2}

\textbf{a) Find with an 8-sample DFT of the unit impulse, \(x(n) = 1, 0, 0, 0, 0, 0, 0, 0, \ldots\).}

\[ X(k) = \sum_{n=0}^{7} x(n) \cdot e^{-j \frac{2\pi}{8}kn} \]

Calculate the 8-sample DFT:
\[ X(k) = 1 + 0 + 0 + \ldots = 1 \]

\textbf{b) Explain the result.}

The result shows that the 8-sample DFT of the unit impulse sequence has a non-zero value only at \(k = 0\). This is expected as the unit impulse has all its energy concentrated at \(n = 0\), and the DFT is essentially calculating the sum of the sequence.

\textbf{c) What result do you expect if you take a 128-sample DFT of the unit impulse sequence?}

\[ \text{Frequency Resolution} = \frac{\text{Sampling Frequency}}{128} \]

For a 128-sample DFT of the unit impulse sequence, the frequency resolution will be \( \frac{2000}{128} \) Hz.

\section*{Problem 3}

\textbf{a) \(X_1(k) = \text{DFT}\{x_1(n)\}\). Find \(X_1(k)\) for the sequence \(x_1(n) = \{2, 2, 0, 0, 2, 2, 0, 0\}\).}

Calculate \(X_1(k)\) for the given sequence:

\[ X_1(k) = \sum_{n=0}^{7} x_1(n) \cdot e^{-j \frac{2\pi}{8}kn} \]
\[ X_1(k) = 2 + 2 + 0 + 0 + 2 + 2 + 0 + 0 = 8 \]

\textbf{b) The sequence \(x_1(n)\) is sampled with a sampling frequency of 1000 Hz. What frequency components in the range \(-\frac{\text{fs}}{2}\) and \(+\frac{\text{fs}}{2}\) are present in \(x(n)\)?}

\[ f = \frac{k \cdot \text{fs}}{N} \]

For the given sequence and sampling frequency:

\[ f = \frac{k \cdot 1000}{8} \]
\[ f = \frac{k \cdot 125}{1} \]

The frequency components present are at multiples of 125 Hz: -500 Hz, -375 Hz, -250 Hz, -125 Hz, 0 Hz, 125 Hz, 250 Hz, 375 Hz.

\textbf{c) Suppose \(x_1(n)\) is a sampled version of \(x_1(t)\) and \(x_1(t)\) is low pass limited to \(\frac{\text{fs}}{2}\). Write the expression for \(x(t)\) from what you found in a) and b) above.}

Since \(x_1(t)\) is low pass limited to \(\frac{\text{fs}}{2}\), the frequency components present in \(x(t)\) are the same as in \(x(n)\). Therefore, \(x(t)\) can be expressed as a sum of sinusoids with frequencies at multiples of 125 Hz.

The expression for \(x(t)\) is given by:
\[ x(t) = \sum_{k=0}^{7} X_1(k) \cdot e^{j \frac{2\pi}{8}kt} \]
Substitute the calculated values of \(X_1(k)\):
\[ x(t) = 8 \cdot e^{j \frac{2\pi}{8} \cdot 0 \cdot t} + 8 \cdot e^{j \frac{2\pi}{8} \cdot 1 \cdot t} + \ldots + 8 \cdot e^{j \frac{2\pi}{8} \cdot 7 \cdot t} \]

Simplify the expression based on the frequencies calculated in part b):

\[ x(t) = 8 + 8 \cdot e^{j \frac{2\pi}{8} \cdot 1 \cdot t} + \ldots + 8 \cdot e^{j \frac{2\pi}{8} \cdot 7 \cdot t} \]

This expression represents the sampled signal \(x_1(n)\) in the continuous-time domain.

\end{document}
