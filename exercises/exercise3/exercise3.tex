\documentclass[12pt]{article}
\usepackage{amsmath}
\usepackage{amsfonts}
\usepackage{amssymb}

\title{Discrete Mathematics Assignment 3}
\author{Gormery K. Wanjiru}
\date{\today}

\begin{document}
\maketitle

\section*{Problem 1}
How many solutions are there to the equation \(x_1 + x_2 + x_3 + x_4 = 10\) if \(x_1, x_2, x_3, x_4\) are nonnegative integers?

\textbf{Solution:}
This problem is solved using the stars and bars theorem. The formula to find the number of nonnegative integer solutions is given by:
\[ \binom{n + k - 1}{k - 1} \]
where \(n = 10\) and \(k = 4\).
\[ \binom{10 + 4 - 1}{4 - 1} = \binom{13}{3} = 286 \]

\section*{Problem 2}
Three officers—a president, a treasurer, and a secretary—are to be chosen from among four people: Ann, Bob, Cyd, and Dan. Bob is not qualified to be treasurer and Cyd is not qualified to be secretary. How many ways can the officers be chosen?

\textbf{Solution:}
\[ \text{Choices for president} = 4 \]
\[ \text{Choices for treasurer (excluding Bob)} = 3 \]
\[ \text{Choices for secretary (excluding Cyd)} = 2 \]
\[ \text{Total ways} = 4 \times 3 \times 2 = 24 \]

\section*{Problem 3}
Find the minimum number of students needed to guarantee that 10 of them were born on the same day of the week.

\textbf{Solution:}
By the pigeonhole principle:
\[ \text{Minimum students} = 7 \times 9 + 1 = 64 \]

\section*{Problem 4}
Given sets \(A, B \subseteq U\), \(A \cap B = \varnothing\), \(n(A) = 12\), \(n(B) = 10\), what is the probability that the selection contains 4 elements from \(A\) and 3 from \(B\)?

\textbf{Solution:}
\[ \text{Favorable outcomes} = \binom{12}{4} \times \binom{10}{3} \]
\[ \text{Total outcomes} = \binom{22}{7} \]
\[ \text{Probability} = \frac{\text{Favorable outcomes}}{\text{Total outcomes}} = \frac{\binom{12}{4} \times \binom{10}{3}}{\binom{22}{7}} \approx 0.3483 \]

\section*{Problem 5}
How many five-person teams contain at most one man from a group of five men and seven women?

\textbf{Solution:}
\[ \text{Teams with no men} = \binom{7}{5} \]
\[ \text{Teams with one man} = \binom{5}{1} \times \binom{7}{4} \]
\[ \text{Total teams} = \binom{7}{5} + \binom{5}{1} \times \binom{7}{4} = 196 \]

\section*{Problem 6}
How many pairs of two distinct integers from the set \(\{1, 2, 3, ..., 100, 101\}\) have a sum that is even?

\textbf{Solution:}
\[ \text{Even pairs (from even numbers)} = \binom{50}{2} \]
\[ \text{Odd pairs (from odd numbers)} = \binom{51}{2} \]
\[ \text{Total even pairs} = \binom{50}{2} + \binom{51}{2} = 2500 \]

\section*{Problem 7}
Eight people are attending the movies together, but two of them do not want to sit next to each other. How many ways can they be seated?

\textbf{Solution:}
\[ \text{Total arrangements} = 8! \]
\[ \text{Unacceptable arrangements (two specific people next to each other). I count them as a unit} = 2 \times 7! \]
\[ \text{Acceptable arrangements} = 8! - 2 \times 7! = 30240 \] Its 2 times the Unacceptable arrangements because the two can switch positions. Albeit sit together in two different ways

\end{document}
