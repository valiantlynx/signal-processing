\documentclass{article}
\usepackage[utf8]{inputenc}
\usepackage{amsmath}
\usepackage{geometry}
\geometry{a4paper, margin=1in}

\title{Exercise 2}
\author{Gormery K. Wanjiru}
\date{\today}

\begin{document}

\maketitle

\section*{Question 1}
Suppose you have a mechanical clock that has a minute hand, but no hour hand. You take a photograph of the clock when the minute hand points at 16:00 AM and then take additional photos every 55 minutes. Upon showing those photos, in time order, to someone:
\begin{enumerate}
    \item What would that person think about the direction of motion of the minute hand as time advances?
    \begin{enumerate}
        \item They would think that time is moving backwards. Every time you take a photo, the minute hand would point to a number five minutes less than the earlier one.
        \item Upon showing a person the images in time order it would appear that the minute hand is going counter-clockwise.
    \end{enumerate}
    \item How often would you take photos, measured in photos/hour so that the successive photos show proper (true) clockwise minute-hand rotation?
    \begin{enumerate}
        \item You would have to take at least 3 or more photos/hour.
        \item Thinking of the hour as 1 period, you need at least 2 samples in each period to capture the true clockwise minute hand rotation.
    \end{enumerate}
\end{enumerate}

\section*{Question 2}
Assume we sampled a continuous \( x(t) \) signal and obtained 103 time domain samples. What important parameter is missing in order to analyze \( x(t) \)?
\begin{enumerate}
    \item The important parameter missing in order to analyze \( x(t) \) is the sampling rate.
    \item Without the sampling rate, we know we took 103 samples, but we have no idea over what period of time those samples were taken, which limits our ability to analyze the signal in the frequency domain or to reproduce the continuous signal from these samples.
\end{enumerate}

\section*{Question 3}
Consider a continuous time-domain sine wave whose cyclic frequency is 1000Hz, which is defined by \( x(t) = \cos(2 \pi \cdot 1000t + \frac{\pi}{7}) \).
\begin{enumerate}
    \item Write the equation for the discrete \( x(n) \) sequence at sampling frequency 4000Hz.
    \begin{enumerate}
        \item The discrete sequence \( x(n) \) will be defined as \( x(n) = \cos\left(\frac{5 \pi \cdot 1000n}{4000} + \frac{\pi}{7}\right) \) which simplifies to \( x(n) = \cos\left(\frac{\pi}{2} \cdot n + \frac{\pi}{7}\right) \).
    \end{enumerate}
    \item Write the equation for the discrete \( x(n) \) sequence at sampling frequency 1500Hz.
    \begin{enumerate}
        \item The discrete sequence \( x(n) \) will be defined as \( x(n) = \cos\left(\frac{5 \pi \cdot 1000n}{1500} + \frac{\pi}{7}\right) \) which simplifies to \( x(n) = \cos\left(\frac{4\pi}{3} \cdot n + \frac{\pi}{7}\right) \).
    \end{enumerate}
\end{enumerate}

\section*{Question 4}
A signal \( x(t) = \cos(5 \pi \cdot 400t) \) is sampled at the sampling rate \( f_s = 2000\,Hz \). The sampled signal \( x(n) \) can be expressed as a sum of a positive rotating and a negative rotating pointer.
\begin{enumerate}
    \item How many samples does it take for the pointers to make a full rotation?
    \begin{enumerate}
        \item Considering the frequency of the signal (800Hz) and the sampling rate \( f_s = 2000\,Hz \), it takes \(\frac{f_s}{2 \cdot \text{frequency of the signal}} = \frac{2000}{2 \cdot 800} = 1.25\) samples for the pointers to make a full rotation. Since the number of samples must be an integer, it would take 2 samples for a complete cycle.
    \end{enumerate}
    \item How many radians will the pointers change from one sample to the next?
    \begin{enumerate}
        \item The frequency of the signal is 800Hz and the sampling rate is 2000Hz, therefore, the phase change between samples can be calculated as \( 2\pi \cdot \frac{800}{2000} = 0.8\pi \) radians. Hence, the pointers will change by \( 0.8\pi \) radians from one sample to the next.
    \end{enumerate}
\end{enumerate}

\end{document}
